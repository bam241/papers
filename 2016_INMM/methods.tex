\section{The \Cyclus Fuel Cycle Simulator}
\label{s_methods}

The \gls{CNERG}\footnote{http://cnerg.github.io/} group at the University of Wisconsin has developed the \Cyclus\footnote{http://fuelcycle.org/} nuclear fuel cycle simulator to model all aspects of the nuclear fuel cycle in a flexible way \cite{cyclus_v1_3}. \Cyclus tracks material flow over time for every agent in a fuel cycle (such as a mine, a nuclear reactor, or even a governing body.  Each agent in the simulation is self-contained and may include physics, economics, or behavioral components \cite{huff_open_2011,gidden_agent-based_2013,gidden_agent-based_2015}.  The agents interact with one another through the \gls{DRE}, which facilitates the trading of resources and commodities \cite{gidden_agent-based_2014}.  At each timestep, agents can choose to request resources, defined using both a quantity and quality (e.g. 1 metric ton of natural uranium with a composition of 99.7\% $^{238}U$ and 0.3\% $^{235}U$).  The \gls{DRE} then solicits bids from any facilities that are interested in offering those resources.  After the bids are received, the requestor is able to state a preference for one bid over another. Finally, the \gls{DRE} calculates all potential trades across all agents, executes an optimization algorithm to find the solution that most closely fulfills a maximum of requests, and the material is transfered across the facilities.  The output of a \Cyclus simulation is a database with information on facility flow and inventories, material composition, transactions between facilities, and facility build and decommissioning histories.

\Cyclus has three key features: it is \textit{agent-based}, it tracks \textit{discrete materials}, and it incorporates \textit{social and behavioral interaction models}. Agent-based design allows for modular simulations where individual facilities can be swapped compared in otherwise identical simulations\cite{jennings_agent-based_2000, taylor2014agent}. \Cyclus tracks discrete material flow through the simulation, and uses data from PyNE,\footnote{http://pyne.io/} to track decay and transumutation data at all timesteps \cite{Scopatz2012b, huff_integrated:_2013}. Finally, behavioral modeling allows facilities and institutions to engage in dynamic decision-making based on their preferences, needs, or political constraints.  A specific agent might have preferences based on material composition, physical proximity between facilities, or allowed and disallowed trading partners, which are implemented in a region-institution-facility hierarchy that enables economic modeling \cite{oliver_geniusv2:_2009}.

Although the primary use case for \Cyclus is to assess the transition from once-through fuel cycles to alternative next-generation scenarios including technologies such as spent fuel recycling, it is was also designed for examining proliferation issues, due to its ability to capture agent behavior. For example, an enrichment facility receiving illicit requests for \gls{HEU} may define its own criteria for whether or not to fulfill this order.  It may disallow production of enrichments above a certain level, choose to trade only with specific facilities, or choose to preferentially fulfill requests at one enrichment level over an other.  Likewise, a requestor of \gls{HEU} can make requests at regular or random intervals, and may request randomly or gaussian distributions of material quantity.  At the insitution level, \Cyclus allows trading decisions between facilities to be controlled by the owner of those facilities, which may be a commercial entity or a nation-state.  This facilitiates modeling of the interactions between multiple states within a region.  






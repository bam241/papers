\section{A Diversion Scenario: Highly Enriched Uranium}
\label{s_results}


% TODO: Actual #s: qty HEU, qty LEU, Power units, gaussian params etc.

% TODO: DIVERSION SCENARIO LAYOUT
\begin{figure}%[htbp!]
\begin{center}
\includegraphics[natwidth=162bp,natheight=227bp, scale=0.6]{./figs/**.png}
\end{center}
\caption{TODO:CAPTION}
\label{fig:scenario_layout}
\end{figure}


As a part of the \gls{CVT}\footnote{http://cvt.engin.umich.edu/}, \Cyclus is being used to generate multi-modal datasets with signatures of diversion with the goal of improving anomaly detection techniques. In the simplest implementation, \gsls{HEU} is clandestinely produced and then diverted from an enrichment facility.  Figure \ref{fig:scenario_layout} illustrates a toy model of this portion of the fuel cycle. A facility such as a mine supplies natural uranium (0.7\% $^{235}$U) to an enrichment facility. The enrichment facility in turn receives requests for \gls{LEU} of various enrichments from several declared light-water reactors (ignoring the fuel fabrication facility for simplicity).  The enrichment facility also receives requests for 90\% enriched \gls{HEU} from an undeclared actor seeking to build a nuclear weapon. Material production for each facility is calculated once each month for a total of 100 months. At each timestep, the enrichment facility fulfills an order for one \gls{LEU} enrichment level, and sometimes produces small quantities of \gls{HEU} request. 

% TODO: Double-plot: Power consumption and LEU production (with HEU production echoed on plot)
\begin{figure}%[htbp!]
\begin{center}
\includegraphics[natwidth=162bp,natheight=227bp, scale=0.6]{./figs/**.png}
\end{center}
\caption{TODO:CAPTION}
\label{fig:time_series}
\end{figure}

Figure \ref{fig:time_series} shows the time series data for declared production of \gls{LEU} (top) and the total SWU consumed by the enrichment plant (bottom) available to the inspector (SWU can be used as a rough proxy for power consumption).  Months where \gls{HEU} is produced are denoted with green on the \gls{LEU} plot.  The \gls{HEU} signature is hidden in the material flow data because there is a gaussian variation (TODO: PARAMS) in the tails assay.  Therefore it is not possible to detect diversion from the individual time-series data alone. However, the two signals can be combined to highlight the correlated signature of diverion. Figure \ref{fig:power_qty} shows time-series data for the ratio of power-consumption to declared \gls{LEU} production.  When the variation due to tails assay (``noise'') is sufficiently small, the deviations can be seen by eye (TODO: FOR EXAMPLE, AT T= ???).

% TODO: Ratio of LEU/POWER showing clear signature of diversion
\begin{figure}%[htbp!]
\begin{center}
\includegraphics[natwidth=162bp,natheight=227bp, scale=0.6]{./figs/**.png}
\end{center}
\caption{TODO:CAPTION}
\label{fig:time_series}
\end{figure}

While this example is clearly a toy model, it illustrates the power of combining multiple signals from a scenario to improve detection capabilities. In practice, as the noise increases or the \gls{HEU} quantity reduces in amplitude, this signature quickly becomes difficult to detect by eye. An important application of \Cyclus is to produce more complex synthetic datasets which are then provided to groups that specialize in developing advanced anomaly detection techniques. An ongoing collaboration with researchers at the Michigan Institute for Data Science\footnote{http://midas.umich.edu/} seeks to apply innovative anomaly detection techniques to these simulations to investigate detection limits for scenarios with sparse data sets or low signal-to-noise ratios.


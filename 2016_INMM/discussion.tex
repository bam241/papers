\section{Discussion}
\label{s_dis}

\Cyclus models signatures of diversion from a diverse set of facilities in the nuclear fuel cycle and with a variety of data modalities listed in  Table \ref{tab:modalities}.  One modality with diverse set of potential applications is satellite imagery.  We are now developing the software infrastructure to create synthetic satellite images with signatures of diversion. Satellite imagery has a variety of applications: tracking personnel or truck movement patterns, thermal or visible signatures of effluent or heat, or major facility changes such as new or removed buildings.

\begin{table}
\centering
\begin{tabular}{|c|c|}
\hline
\textbf{Modality}        & \textbf{Physical Signal} \\
\hline
Time-series              & Material Flow \\
                         & Power Usage \\
\hline
Discrete Event           & Shipments \\
\hline
Sparse Data              & Inspections \\
\hline
Geospatially Distributed & Effluent Dispersion  \\
\hline
Energy-series            & Neutron Spectra \\
\hline
Surveillance Images      & Resource tracking \\
(under development)      & Thermal Maps  \\
                         & Infrastructure Modifications \\
\hline
\end{tabular}
\caption{Data modalities available in \Cyclus and their fuel cycle applications}
\label{tab:modalities}
\end{table}

These diverse datasets can be combined to highlight signatures of diversion that are small enough to be hidden in the noise of individual signals.  We have illustrated this technique by combining time-series data for power consumption and declared \gls{LEU} production for a simple scenario of \gls{HEU} production in an enrichment facility.  More realistic scenarios require advanced anomaly detection techniques such as those being developed at \gls{UM}. A collaboration with \gls{UM} and \gls{Sandia} is planned to investigate ways to optimize subsets of diverse signal modalities to ensure reliable detection while minimizing resource usage. Additionally, \Cyclus has the capability to produce synthetic signals of inherent physical processes such as neutron spectra of various materials.  In this way, \Cyclus simulations can provide theoretical signals to researchers developing experimental detectors in order to test sensitivity and detector response. The \Cyclus fuel cycle simulator is being used as a framework for combining techniques and knowledge from a variety of disciplines to support a cohesive approach to treaty verification. 


\textit{This work was funded by the Consortium for Verification Technology under Department of Energy National Nuclear Security Administration award number DE-NA0002534”}

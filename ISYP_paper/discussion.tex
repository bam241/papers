\section{Discussion}
\label{s_dis}

The \Cyclus fuel cycle simulator is being used as a framework for combining techniques and knowledge from a variety of disciplines into a cohesive treaty verification approach. This paper presents the diversion of \gls{HEU} as a simple example of measuring facility output as a treaty verification strategy.  This \gls{HEU} simulation used a random-interval, constant-amplitude model to describe the behavior of an illicit actor seeking to divert \gls{HEU} from a declared enrichment facility.  It uses statistical anomaly detection techniques on the data that would be available to an \gls{IAEA} inspector, namely the \gls{LEU} output of the enrichment facility, to demonstrate that material diversion could be detected in this scenario.

This simulation is but one example of the capabilities of \Cyclus as a test bed for treaty verification strategies.  \Cyclus is also being used to model the geographical dissipation of $^{85}Kr$ emission from a covert separations facility processing \gls{WGP} in the effluent shadow of declared facilities.  This feature can be used to assess the required regional distribution of various types of $^{85}Kr$ detectors to ensure detection of clandestine reprocessing programs.  Additionally, \Cyclus has the capability to produce synthetic signals of inherent physical processes such as neutron spectra of various materials.  In this way, \Cyclus simulations can provide theoretical signals to researchers developing experimental detectors in order to test sensitivity and detector response.

Moving forward, \Cyclus will be used to study more complex diversion scenarios.  Probabilistic models for behavior based on risk-perception will be explored.  Ongoing collaborations as part of the \gls{CVT} are examining the mechanisms and limits of expanding anomaly detection algorithms with other types of data, such as social media chatter.  Due to the inherently interdisciplinary nature of this work, new external collaborations are sought with experts in behavioral modeling. Innovative ideas on detection modalities and diversion detection techniques are also welcomed.



\textit{\TODO{This work is funded by the NNSA as part of the Consortium for Verification Technology.  . ..}}
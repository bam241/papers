\section{Discussion}
\label{s_dis}

The \Cyclus fuel cycle simulator can be used as a framework for combining techniques and knowledge from a variety of disciplines into a cohesive treaty verification scenario. This paper presents the diversion of \gls{HEU} as a simple example of measuring facility throughput as a treaty verification strategy.  The \gls{HEU} simulation used a random time-base, constant amplitude model to describe the behavior of an illicit actor seeking to divert \gls{HEU} from a declared enrichment facility.  It applied a statistical anomaly detection technique to the data that would be available to an \gls{IAEA} inspector, namely the \gls{LEU} output of the enrichment facility to demonstrate that material diversion could be detected in this scenario.

This simulation is but one example of the capabilities of \Cyclus as a test bed for verification strategies.  \Cyclus is also being used to model the geographical dissipation of $^{85}Kr$ emission from a covert separations facility processing \gls{WGP} in the shadow of declared facilities.  This feature can be used to assess the desired regional distribution of $^{85}Kr$ detectors to ensure detection of clandestine reprocessing programs.  Additionally, \Cyclus has the capability to produce synthetic signals of inherent physical processes such as neutron spectra of various materials.  As a result, \Cyclus simulations can provide theoretical signals to researchers developing experimental detectors in order to test sensitivity and detector response.

Moving forward, \Cyclus will be used to study more complex simulation scenarios.  Probabilistic models for behavior based on risk-perception will be explored.  Ongoing collaborations as part of the \gls{CVT} will examine the mechanisms and limits of expanding anomaly detection algorithms to scenarios with diverse detection modalities and sparse datasets.  Due to the inherently interdisciplinary nature of this work, external collaborations are sought in particular with experts in behavioral models of illicit actors. Innovative ideas on detection modalities and diversion detection techniques are also welcomed.



\textit{\TODO{This work is funded by the NNSA as part of the Consortium for Verification Technology.  . ..}}
\section{The \Cyclus Fuel Cycle Simulator}
\label{s_methods}

The \gls{CNERG}\footnote{http://cnerg.github.io/} group at the University of Wisconsin has developed the \Cyclus\footnote{http://fuelcycle.org/} nuclear fuel cycle simulator to model all aspects of the nuclear fuel cycle in a flexible way\cite{cyclus_v1_0,cyclus_v1_2,cyclus_v1_3}.  \Cyclus tracks nuclear material as it flows through the entire fuel cycle in discrete timesteps. It has been designed to compare different types of technologies in the transition from once-through cycles to alternative next-generation scenarios including technologies such as spent fuel recycling. \Cyclus features a modular design in which individual agents can be swapped in otherwise identical fuel cycles for comparison. For example, a user could compare enrichment technologies by creating two different enrichment agents, one using gaseous diffusion and the other using centrifuge technology. Then two simulations can be run where the entire fuel cycle is identical except for the two different enrichment designs, and the results can therefore be directly compared.  \Cyclus has three key features that make it well-suited to non-proliferation studies: it is \textit{agent-based}, it incorporates \textit{social and behavioral interaction models}, and it tracks \textit{discrete materials}.

\subsection{Agent-Based}
\Cyclus is designed using an agent-based framework, meaning that each actor in a fuel cycle (such as a mine, a nuclear reactor, or even a governing body) is modeled as an independent agent\cite{jennings_agent-based_2000, taylor2014agent}. Each agent is self-contained and may include physics, economics, or behavioral components\cite{huff_open_2011,gidden_agent-based_2013,gidden_agent-based_2015}.  The agents interact with the other agents in the fuel cycle through the \gls{DRE}, which facilitates the trading of resources and commodities\cite{gidden_agent-based_2014}.  At each timestep, agents can choose to request resources.  Resources are defined using both a quantity (e.g. 1 metric ton), and a quality, such as having a composition of 99.7\% $^{238}U$ and 0.3\% $^{235}U$.  The \gls{DRE} then solicits bids from any facilities that are interested in offering those resources. Resources can be offered as bids even if they do not exactly match the requested material. For example, a reactor might request a commodity called ``fuel'', which it has defined as being \gls{UOX}.  It may receive two bids for ``fuel'' that are specified as \gls{UOX} and \gls{MOX}, having two distinct isotopic compositions. After the bids are received, the requestor is able to state a preference for one bid over another. Finally, once the preferences have been applied, the \gls{DRE} calculates all potential trades across all agents, then executes an optimization algorithm to find the solution that most closely fulfills a maximum of requests. It is possible that as a result of the preference adjustment, no trades are executed for some facilities in a given time step.  Once this solution is found, material is transfered across the facilities and the timestep is concluded.

\subsection{Behavioral Modeling}
The preference adjustment phase of the \gls{DRE} allows for the introduction of interaction behaviors.  Each agent can prioritize bids for resources in any way it chooses. A specific agent might have preferences based on material composition, physical proximity between facilities, or allowed and disallowed trading partners. For example, \Cyclus has a region-institution-facility hierarchy that enables economic modeling\cite{oliver_geniusv2:_2009}.  Individual facilities can be managed by insistutions, such as multinational corporations, utilities, government agencies, etc.  Additionally, facilities and institutions can be ascribed to distinct regions, which may represent geographical regions, political alliances or economic trading partners. This feature allows \Cyclus to model tariffs, sanctions and other types of economic agreements.    Agents may also make decisions about interacting at each timestep based on their own internal logic, for example, an illicit facility may choose not to trade at every timestep in an attempt to avoid detection. 

\subsection{Discrete Materials}
\Cyclus tracks discrete material flow through the simulation, meaning that once a material enters the simulation, its location and quality is tracked at all remaining timepoints\cite{huff_integrated:_2013}.  \Cyclus includes nuclear data from \emph{Pyne},\footnote{http://pyne.io/} a computational nuclear science tool that enables calculation of decay, transmutation, diffusion and other nuclear physics\cite{Scopatz2012b}. As a result, \Cyclus can  model decay of materials and track all decay products from a parent isotope, facilitating studies of heat loading, radiation exposure, and other derived fuel cycle metrics\cite{scopatz_cymetric_2015}. This also provides material attribution capability, with applications in nuclear forensics and archeology.

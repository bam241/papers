\section{Introduction}
\label{s_motive}



%An effective treaty verification regime must synthesize knowledge from the realms of political science, international relations, nuclear physics and engineering, and even behavioral psychology.  Figure \ref{fig:cyclus_diagram} illustrates  the role of a fuel cycle simulator such as \Cyclus in bringing together these disparate fields to provide insights into proposed verification technologies. A fuel cycle simulator tracks the flow of nuclear material through the facilities in a fuel cycle\cite{gidden_agent-based_2015}.  Uranium enrichment and spent fuel reprocessing are two particularly sensitive parts of the fuel cycle, but correlated signatures of illicit activity are likely to be present across the fuel cycle. A fuel cycle simulator creates synthetic data, such as what would be available to an inspector, for many different facilities simultaneously while incorporating a system-level perspective of proliferation scenarios. This synthetic data can then be used as a testbed to investigate the efficacy of new detection and analysis techniques and illucidate the strengths and weaknesses of various verification strategies.


Progress in preventing the spread of nuclear weapons is contigent both upon preventing access to the required technology as well as disincentivizing weapons programs. An effective global nonproliferation regime must synthesize knowledge from the realms of political science, international relations, nuclear physics and engineering, and even behavioral psychology. An understanding of when and why states proliferate can inform future treaties and agreements on the policy side to minimize this risk. For example, the concept of multilateral enrichment, in which multiple states co-own and operate an enrichment facility, has the potential to reduce the spread of enrichment technology. However, detractors point to the improved international networking opportunities inherent in multinational organizations as a risk factor for increased proliferation. A regional or global framework to compare the relative risk between proposed treaty paradigms and the status quo can help inform the discussion and potentially identify ways to reduce global risk of nuclear proliferation.

As part of the Consortium for Verification Technology\TODO{CITE}, the Cyclus fuel cycle simulator is being used as a test-bed for the development of such new technologies and approaches to treaty development. Cyclus is a systems-level nuclear fuel cycle simulator that models the interactions between actors in the nuclear arena. While designed to track the flow of nuclear material between facilities, Cyclus also incorporates an innovative Region-Institution-Facility hierarchy that can capture the dynamics of regional-level state interactions.  We have developed a forward model in which a set of hypothetical states can be given individual political and technical attributes. This model then tracks whether each state develops nuclear weapons over time based on how these attributes evolve in time, including feedback on conflict levels between states.

